\documentclass[journal]{IEEEtran}
\IEEEoverridecommandlockouts
\usepackage{graphicx} 
\usepackage{amsmath,amssymb,amsfonts,xcolor,amsthm}
\usepackage{soulutf8}
\usepackage{amsmath,amssymb,amsfonts,xcolor,amsthm}
\usepackage{algorithm,algorithmic}
\usepackage{stfloats}
\usepackage{cite}
\usepackage{float}
\usepackage{graphicx}         
\usepackage{textcomp}
\usepackage[export]{adjustbox}
\usepackage{longtable}
\usepackage{balance}
\usepackage{tikz}
\usetikzlibrary{mindmap,trees}
\allowdisplaybreaks
\newtheorem{lemma}{{\bf Lemma}}
\usepackage{tabularx}
\usepackage{boldline}
\usepackage[normalem]{ulem}



\begin{document}

\title{A Tutorial on Near-Field Driven 6G networks}
\author{Arnav Mukhopadhyay, Keshav Singh,
Fan-Shuo Tseng, Sandeep Kumar Singh, and Kapal Dev
\thanks{\hrule}
\thanks{A. Mukhopadhyay, K. Singh, and
F.-S. Tseng are with the Institute of Communications Engineering, National Sun Yat-sen University, Kaohsiung, Taiwan (E-mail: gudduarnav@gmail.com, 
keshav.singh@mail.nsysu.edu.tw, fs.tseng@mail.nsysu.edu.tw).}
\thanks{S. K. Singh is with the Department of ECE, Motilal Nehru National Institute of Technology Allahabad, India (E-mail: sksingh@mnnit.ac.in).}
\thanks{K. Dev is with the Department of Computer Science, Munster Technological University, Ireland (E-mail: kapal.dev@ieee.org).}
%\thanks{C. Pan is with the National Mobile Communications Research Laboratory, Southeast University, Nanjing, China (E-mail: cpan@seu.edu.cn).} 

}

\maketitle

% ===== Abstract =====
\begin{abstract}
Abstact
\end{abstract}

\begin{IEEEkeywords}
key1, key2, key3, key4, key5, key6.
\end{IEEEkeywords}



% ===== Introduction =====
\section{Introduction}
 ---- introduction ----
 
% ===== Conclusion =====
\section{Conclusion}
--- conclusion ---

% ===== end of main contents =====
%\balance
%\nocite{*}
% \bibliographystyle{IEEEtran}
\bibliographystyle{IEEEtranN} 
\bibliography{IEEEabrv, ref}
\end{document}
